\documentclass[a4paper,12pt]{article}


\title{Analyzing Network Transfer Protocols on a Simulated Interplanetary Internet}
\author{Cody Auch, Mark Lysack, Jacob Janzen}
\date{\today}

\begin{document}
\maketitle

\begin{abstract}
    TODO insert abstract here
\end{abstract}

\section{Introduction}

In the past decade interest in returning back to space has increased internationally. Many organizations 
have started sending remote vehicles into space for the first time \cite{(indian space program, spaceX, ect)}
and there are talks of a permanent lunar base being created. With an increased amount of satellites and potentially 
permanent bases outside of Earths orbit, the demand on the interplanetary network (IPN) is increased. The IPN has a 
series of unique challenges such as a high amount of error, constantly changing network topology, and long 
propagation delay. The solution to these problems is implementing a delay tolerant network. While space organizations 
have maintained the IPN with a set of standard protocols, these may not be sufficient for an expanding network.
The simulator proposed shows how an IPN can handle different stages of development of an IPN as well as increased 
throughput requirements.

\section{Previous Work}

\section{Methodology}

\subsection{Physics Engine}

A significant challenge in creating a IPN is the rapidly changing topology. To simulate 
these rapid changing conditions a physics engine was created to model the orbits of several 
entities in the solar system. Several assumptions were made in the simulator for the sake of 
simplicity as the goal of the simulator is not to create a perfect model of the solar system 
and the entities inside of the solar system, but rather conditions which a IPN would exist in. 

The first assumption made was that all orbits are circular. Realistically, orbits are oval 
shaped, with the two focal points. A result of this assumption is the elimination of any 
difference in orbital speed due to Kepler's third law. The orbital radius was chosen by taking
the average orbital radius of the real life celestial bodies.
For example Earth's average orbital
radius is 149.6 billion meters \cite{}, and therefore orbits around the sun in a circular orbit
of radius 149.6 billion meters.  Another result of this assumption 
is that orbits are considered to be centered around the body which it orbits. The Mars orbit 
has a focal point that is further in space, making it more oval shaped than the earth orbit. \cite{} 
While this could make changes to a 
simulated network in some very specific situations, the primary behaviour of a solar system 
is maintained, like retrograde orbits, periodical alignment, predictable behaviour, and 
recursive orbits of sub orbiting object.

All entities are assumed to have a circular shape, with a constant radius. Due to the scale of the 
objects, both large and small, this has a very minor effect.

The whole physics simulation has been done on a 2D plane. While planetary orbits were not perfectly
aligned, on the astronomical scale which we are working on, assuming that all their orbits are 
contained on the same plane will not effect the simulation significantly. Where this could be an issue 
is for when there are more than two satellites orbiting the same entity, with a relatively low orbit. For
example satellites orbiting Earth. While there are some complex satellite interactions which are removed 
it does still simulate satellites having to form a network around the entity it is 
orbiting, it just requires less satellites then if it needed to form a 3D network. 

Messages between entities were assumed to be sent at the speed of light, as all forms of radio, lasers,
or other wireless messages all are sent at the speed of light. Since propagation delay is much higher 
then terrestrial networks, other delays (TODO specify which delays) were ignored as being negligible. 

Based on these simplifications, a physical simulation with several goals. To calculate if a line of 
sight could be made between two communicating entities, that is if a message can be sent, to calculate 
the error rate when sending a message, and to calculate the message propagation delay for the message to 
be sent. Line of sight was achieved using simple geometry calculations. (See appendix) All entities are able to block 
signals if the ray passes through the radius of the entity. (TODO Cody inserts bit about error calculation)
Propagation delay is a simple function of distance and the speed of light. The input of the physics engine 
is a time $t$ from some preset initial conditions which outputs the state of all entities.


\end{document}
