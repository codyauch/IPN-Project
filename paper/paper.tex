\documentclass[a4paper,12pt]{article}


\title{Analyzing Network Transfer Protocols on a Simulated Interplanetary Internet}
\author{Cody Auch, Mark Lysack, Jacob Janzen}
\date{\today}

\begin{document}
\maketitle

\begin{abstract}
    TODO insert abstract here
\end{abstract}

\section{Introduction}

In the past decade interest in returning back to space has increased internationally. Many organizations 
have started sending remote vehicles into space for the first time \cite{(indian space program, spaceX, ect)}
and there are talks of a permanent lunar base being created. With an increased amount of satellites and potentially 
permanent bases outside of Earths orbit, the demand on the interplanetary network (IPN) is increased. The IPN has a 
series of unique challenges such as a high amount of error, constantly changing network topology, and long 
propagation delay. The solution to these problems is implementing a delay tolerant network. While space organizations 
have maintained the IPN with a set of standard protocols, these may not be sufficient for an expanding network.
The simulator proposed shows how an IPN can handle different stages of development of an IPN as well as increased 
throughput requirements. Before these technological changes can be implemented simulations need to be created to 
lower the risk of failure given the high cost of failure. Often satellites will need to be in operation for years 
or even decades. The hardware and software in satellites can often not be updated very well. Thus it will be very 
relevant to investigate these issues well before they are needed. 


\section{Previous Work}

In their 2017 Masters Thesis, Md Monjurul Islam Khan \cite{Khan2017} from University of Manitoba wrote on how 
to create a satellite network
simulation based on ns3. They used a point to point ns3 module to create a network of 2 levels of orbiting 
satellites and fixed points on the earth which acts as uplinks to the network. The paper analyzed TCP 
HighSpeed against TCP Vegas slow start. However this paper stops at expanding the network beyond earth orbit, 
and rather focuses on Earth orbit communication.  

SNS3 is a simulation library which uses NS3 as a basis to create a framework which to build simulations for 
geostationary satellites in the european continent. It does not expand to an interplanetary network.
\cite{10.4108/icst.simutools.2014.254631}

For non simulation work done on the interplanetary internet, the leading authority is Consultative Committee for 
Space Data Systems (CCSDS). \cite{CCSDS.org} They maintain and publish the standards for interplanetary internet, including 
the SCPS TP, a reliable transport protocol used for delay tolerant networks when UDP or TCP are not effective. \cite{40565_2004}
Active since 1982 and managed in collaboration with several national and international space organizations. A 
contemporary simulation of a current interplanetary network or earth orbit network would follow the protocols and
standards laid out by CCSDS.

Several public github repos with satellite simulators can be found. 
$\text{https://github.com/szymonwieloch/DTN/tree/master}$
$\text{https://goodingc.github.io/ipn_sim/web-app/}$. Both of these limit themselves to earth orbit simulations. 

\section{Interplanetary Internet and Background}

An interplanetary network is simply a network which where the nodes are distributed across the surface or orbits of 
different planets. In the year of 2024 the current interplanetary is limited to Mars, Earth, the Moon, and satellites 
orbiting the sun. The interplanetary network is also expanded to deep space satellites, like the voyager probes. A sample 
message path across the network could be that a message is passed from a on earth station through satellite dish to a 
earth orbiting satellite, then when it has direct line of sight with the Mars MAVEN sends the message. MAVEN then relays the 
message to a Mars rover like Odyssey. This description hides the complexity which arises in an interplanetary network. 

In a traditional network a majority of the network will happen over a wired connection. This means that the error is relatively 
low. This is no longer the case in the case of satellites sending messages over potentially millions of kilometers. 
Atmosphere in the Earth orbit, solar radiation, and dust cause increased error in sending the message between entities. 
Inverse square law dictates that over these large periods the energy needed to send the message so that it can be 
read at its destination. Even the doppler effect is needed to be accounted for to tune into radio frequencies on the ISS.
High error means that sending a message can no longer be assumed to be sent reliably and has to be compensated for. 
Beyond high error, large distances between entities and the physical limitation that messages cannot travel faster 
than the speed of light. Light, or any other energy takes 8 minutes to get from the Sun to Earth, and thus for a round trip 
will take 16 minutes to complete a simple handshake, given that the messages were sent correctly.

Other than physical limitations a interplanetary network has a rapidly changing topology at anytime. For instance satellites 
which orbit planets can have their field of view blocked as they move over the horizon. While terrestrial networks also have 
rapidly changing networks at the edges, but the difference is that the core of the network is stable traditionally. Where 
depending on the hour, day, year, the core of the interplanetary network can change drastically. To be able to connect to the 
network there are no well known places to access the network, except for Earth. 

To combat these challenges several protocols have been created. The SCPS TP is a transport protocol which is used on the network 
layer. UDP is used in the lowest reliable connections between nodes, and TCP is used in the most reliable network 
conditions. When a reliable and efficient method of network transportation is needed, then SCPS TP is used. It is a modified 
version of TCP which uses a slow start protocol, reduced handshakes, reliable data transfer with reduced confirmations. Other 
protocols like store and load, and licklider are used to also overcome these problems.

The rest of the paper remains to simulate these unique challenges, as well as see in what conditions do the current protocols 
work.

\section{Methodology}

\subsection{Physics Engine}

A significant challenge in creating a IPN is the rapidly changing topology. To simulate 
these rapid changing conditions a physics engine was created to model the orbits of several 
entities in the solar system. Several assumptions were made in the simulator for the sake of 
simplicity as the goal of the simulator is not to create a perfect model of the solar system 
and the entities inside of the solar system, but rather conditions which a IPN would exist in. 

The first assumption made was that all orbits are circular. Realistically, orbits are oval 
shaped, with the two focal points. A result of this assumption is the elimination of any 
difference in orbital speed due to Kepler's third law. The orbital radius was chosen by taking
the average orbital radius of the real life celestial bodies.
For example Earth's average orbital
radius is 149.6 billion meters \cite{}, and therefore orbits around the sun in a circular orbit
of radius 149.6 billion meters.  Another result of this assumption 
is that orbits are considered to be centered around the body which it orbits. The Mars orbit 
has a focal point that is further in space, making it more oval shaped than the earth orbit. \cite{} 
While this could make changes to a 
simulated network in some very specific situations, the primary behaviour of a solar system 
is maintained, like retrograde orbits, periodical alignment, predictable behaviour, and 
recursive orbits of sub orbiting object.

All entities are assumed to have a circular shape, with a constant radius. Due to the scale of the 
objects, both large and small, this has a very minor effect.

The whole physics simulation has been done on a 2D plane. While planetary orbits were not perfectly
aligned, on the astronomical scale which we are working on, assuming that all their orbits are 
contained on the same plane will not effect the simulation significantly. Where this could be an issue 
is for when there are more than two satellites orbiting the same entity, with a relatively low orbit. For
example satellites orbiting Earth. While there are some complex satellite interactions which are removed 
it does still simulate satellites having to form a network around the entity it is 
orbiting, it just requires less satellites then if it needed to form a 3D network. 

Messages between entities were assumed to be sent at the speed of light, as all forms of radio, lasers,
or other wireless messages all are sent at the speed of light. Since propagation delay is much higher 
then terrestrial networks, other delays (TODO specify which delays) were ignored as being negligible. 

Based on these simplifications, a physical simulation with several goals. To calculate if a line of 
sight could be made between two communicating entities, that is if a message can be sent, to calculate 
the error rate when sending a message, and to calculate the message propagation delay for the message to 
be sent. Line of sight was achieved using simple geometry calculations. (See appendix) All entities are able to block 
signals if the ray passes through the radius of the entity. (TODO Cody inserts bit about error calculation)
Propagation delay is a simple function of distance and the speed of light. The input of the physics engine 
is a time $t$ from some preset initial conditions which outputs the state of all entities.


\end{document}
