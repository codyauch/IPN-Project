\documentclass[a4paper,12pt]{article}

\usepackage{microtype}
\usepackage{biblatex}
\usepackage{hyperref}

\addbibresource{paper.bib}

\title{Analyzing Network Transfer Protocols on a Simulated Interplanetary Internet}
\author{Cody Auch, Mark Lysack, Jacob Janzen}
\date{\today}

\newcommand{\CC}{C\nolinebreak\hspace{-.05em}\raisebox{.4ex}{\tiny\bf +}\nolinebreak\hspace{-.10em}\raisebox{.4ex}{\tiny\bf +}}
\def\CC{{C\nolinebreak[4]\hspace{-.05em}\raisebox{.4ex}{\tiny\bf ++}}}

\begin{document}
\maketitle

\begin{abstract}
    TODO insert abstract here
\end{abstract}

\section{Introduction}

In the past decade interest in returning back to space has increased
internationally. Many organizations have started sending remote vehicles into
space for the first time~%\cite{(indian space program, spaceX, ect)}
and there are talks of a permanent lunar base being created. With an increased
amount of satellites and potentially permanent bases outside of Earths orbit,
the demand on the interplanetary network (IPN) is increased. The IPN has a
series of unique challenges such as a high amount of error, constantly changing
network topology, and long propagation delay. The solution to these problems is
implementing a delay tolerant network. While space organizations have maintained
the IPN with a set of standard protocols, these may not be sufficient for an
expanding network. The simulator proposed shows how an IPN can handle different
stages of development of an IPN as well as increased throughput requirements.
Before these technological changes can be implemented simulations need to be
created to lower the risk of failure given the high cost of failure. Often
satellites will need to be in operation for years or even decades. The hardware
and software in satellites can often not be updated very well. Thus it will be
very relevant to investigate these issues well before they are needed.


\section{Previous Work}

In their 2017 Masters Thesis, Md Monjurul Islam Khan~\cite{Khan2017} from
University of Manitoba wrote on how to create a satellite network simulation
based on ns3. They used a point to point ns3 module to create a network of 2
levels of orbiting satellites and fixed points on the earth which acts as
uplinks to the network. The paper analyzed TCP HighSpeed against TCP Vegas slow
start. However this paper stops at expanding the network beyond earth orbit, and
rather focuses on Earth orbit communication.

SNS3 is a simulation library which uses NS3 as a basis to create a framework
which to build simulations for geostationary satellites in the European
continent. It does not expand to an interplanetary network~\cite{Puttonen2014}.

For non simulation work done on the interplanetary internet, the leading
authority is Consultative Committee for Space Data Systems
(CCSDS)~\cite{CCSDS.org}. They maintain and publish the standards for
interplanetary internet, including the SCPS TP, a reliable transport protocol
used for delay tolerant networks when UDP or TCP are not
effective.~\cite{Keith2004} Active since 1982 and managed in collaboration with
several national and international space organizations. A contemporary
simulation of a current interplanetary network or earth orbit network would
follow the protocols and standards laid out by CCSDS.

Several public GitHub repositories with satellite simulators can be
found, for instance:
\begin{itemize}
\item\url{https://github.com/szymonwieloch/DTN/tree/master}
\item\url{https://goodingc.github.io/ipn_sim/web-app/}
\end{itemize}
Both of these limit themselves to earth orbit simulations.

\section{Interplanetary Internet and Background}

An interplanetary network is simply a network which where the nodes are
distributed across the surface or orbits of different planets. In the year of
2024 the current interplanetary is limited to Mars, Earth, the Moon, and
satellites orbiting the sun. The interplanetary network is also expanded to deep
space satellites, like the voyager probes. A sample message path across the
network could be that a message is passed from a on earth station through
satellite dish to a earth orbiting satellite, then when it has direct line of
sight with the Mars MAVEN sends the message. MAVEN then relays the message to a
Mars rover like Odyssey. This description hides the complexity which arises in
an interplanetary network.

In a traditional network a majority of the network will happen over a wired
connection. This means that the error is relatively low. This is no longer the
case in the case of satellites sending messages over potentially millions of
kilometers. Atmosphere in the Earth orbit, solar radiation, and dust cause
increased error in sending the message between entities. Inverse square law
dictates that over these large periods the energy needed to send the message so
that it can be read at its destination. Even the Doppler effect is needed to be
accounted for to tune into radio frequencies on the ISS. High error means that
sending a message can no longer be assumed to be sent reliably and has to be
compensated for. Beyond high error, large distances between entities and the
physical limitation that messages cannot travel faster than the speed of light.
Light, or any other energy takes 8 minutes to get from the Sun to Earth, and
thus for a round trip will take 16 minutes to complete a simple handshake, given
that the messages were sent correctly.

Other than physical limitations a interplanetary network has a rapidly changing
topology at anytime. For instance satellites which orbit planets can have their
field of view blocked as they move over the horizon. While terrestrial networks
also have rapidly changing networks at the edges, but the difference is that the
core of the network is stable traditionally. Where depending on the hour, day,
year, the core of the interplanetary network can change drastically. To be able
to connect to the network there are no well known places to access the network,
except for Earth.

To combat these challenges several protocols have been created. The SCPS TP is a
transport protocol which is used on the network layer. UDP is used in the lowest
reliable connections between nodes, and TCP is used in the most reliable network
conditions. When a reliable and efficient method of network transportation is
needed, then SCPS TP is used. It is a modified version of TCP which uses a slow
start protocol, reduced handshakes, reliable data transfer with reduced
confirmations. Other protocols like store and load, and Licklider are used to
also overcome these problems.

The rest of the paper remains to simulate these unique challenges, as well as
see in what conditions do the current protocols work.

\section{Methodology}

\subsection{Physics Engine}

A significant challenge in creating a IPN is the rapidly changing topology. To
simulate these rapid changing conditions a physics engine was created to model
the orbits of several entities in the solar system. Several assumptions were
made in the simulator for the sake of simplicity as the goal of the simulator is
not to create a perfect model of the solar system and the entities inside of the
solar system, but rather conditions which a IPN would exist in.

The first assumption made was that all orbits are circular. Realistically,
orbits are oval shaped, with the two focal points. A result of this assumption
is the elimination of any difference in orbital speed due to Kepler's third law.
The orbital radius was chosen by taking the average orbital radius of the real
life celestial bodies. For example Earth's average orbital radius is 149.6
billion meters~\cite{}, and therefore orbits around the sun in a circular orbit
of radius 149.6 billion meters. Another result of this assumption is that orbits
are considered to be centered around the body which it orbits. The Mars orbit
has a focal point that is further in space, making it more oval shaped than the
earth orbit.~\cite{} While this could make changes to a simulated network in
some very specific situations, the primary behaviour of a solar system is
maintained, like retrograde orbits, periodical alignment, predictable behaviour,
and recursive orbits of sub orbiting object.

All entities are assumed to have a circular shape, with a constant radius. Due
to the scale of the objects, both large and small, this has a very minor effect.

The whole physics simulation has been done on a 2D plane. While planetary orbits
were not perfectly aligned, on the astronomical scale which we are working on,
assuming that all their orbits are contained on the same plane will not effect
the simulation significantly. Where this could be an issue is for when there are
more than two satellites orbiting the same entity, with a relatively low orbit.
For example satellites orbiting Earth. While there are some complex satellite
interactions which are removed it does still simulate satellites having to form
a network around the entity it is orbiting, it just requires less satellites
then if it needed to form a 3-D network.

Messages between entities were assumed to be sent at the speed of light, as all
forms of radio, lasers, or other wireless messages all are sent at the speed of
light. Since propagation delay is much higher then terrestrial networks, other
delays (TODO specify which delays) were ignored as being negligible.

Based on these simplifications, a physical simulation with several goals. To
calculate if a line of sight could be made between two communicating entities,
that is if a message can be sent, to calculate the error rate when sending a
message, and to calculate the message propagation delay for the message to be
sent. Line of sight was achieved using simple geometry calculations. (See
appendix) All entities are able to block signals if the ray passes through the
radius of the entity. One of the largest factors contributing to transmission
error is the distance travelled by the electromagnetic signal between
transmitting and receiving nodes. By generalizing Friis' transmission formula
for isotropic emitters and receivers, we can obtain the ratio of signal
intensity received to signal intensity transmitted, known as Free Space Path
Loss (FSPL). We can use FSPL as our transmission error rate, giving the
following expression: $Error Rate = {(\frac{4 \pi d}{\lambda})}^2$ where $d$ is
the distance between nodes in metres and $\lambda$ is the wavelength of the
electromagnetic wave used in transmission. As much of space communications
utilizes the Ka-band of frequencies (27--40 GHz) we opted to use a transmission
frequency of 30Ghz. The transmission wavelength can be then calculated according
to $\lambda = \frac{c}{f}$, where $c$ is the speed of light and $f$ is the
transmission frequency, giving a transmission wavelength of about 9.99~mm.
Propagation delay is a simple function of distance and the speed of light. The
input of the physics engine is a time $t$ from some preset initial conditions
which outputs the state of all entities.

\subsection{Network Simulation}

We used NS-3~\cite{ns-3} to implement a simulation of this network. Our final design
consisted of a set of routers representing the different entities in our physics
simulation which were each linked to every other router using a point-to-point
channel. Essentially, this allowed us to have a wired network with configurable
delay and error along with the ability to take certain channels down if they are
obstructed that we could make behave like an interplanetary network would
without having to deal with the complexities of configuring wireless
communication. In addition to the mesh of routers, we added two individual nodes
to the network with zero delay that can be connected to any one router in the
network. One of these two nodes acts as a sender and the other one acts as a
receiver. This configuration of sender and receiver as separate nodes made it
easier to decouple the applications from the network topology and allowed us to
programatically create arbitrary topologies.

NS-3 is a useful library, but we did run into a number of problems over the
course of our implementation of our simulation. The first problem came when we
first tried to compile the library itself and ran into a bug in the build script
that caused all executables to fail to run if there was the word ``scratch''
anywhere in the path due to a specific subdirectory of NS-3 which also happens
to be called ``scratch''. When first testing it out, one of us had put NS-3 into
a directory that happened to contain that word. We made a pull request to fix
this issue and it was merged by the NS-3 maintainers which allowed us to
continue.

The next issue we ran into was that we had decided to use the Python bindings to
allow us to easily use Python for the physics simulation. This was a mistake.
Although NS-3 does have Python bindings that do make it easier to run it as a
script after writing, there is no documentation at all for the Python bindings
and some basic features such as setting the error rate on a channel or
scheduling a task to run during the simulation have no Python support and must
use inline \CC. The lack of support for task scheduling in Python is
particularly bad as it forced us to schedule a \CC{} function which itself
called Python code. It also forced us to use global state as \texttt{CPyCpyy}
does not have support for arbitrary function parameters when calling Python code
as far as we could tell (this feature is not documented and we only figured out
about it because the few Python examples in the NS-3 source code that do use
scheduling also call Python from \CC{} with this method).

Before we settled on using point-to-point to emulate a wireless network, we
attempted to use the existing wireless networking features of NS-3. These
features are very powerful and even have a positioning and velocity system for
nodes in the network which would be very useful. We ran into a lot of issues
with this system though. Firstly, there only seems to be real support for Wi-Fi
which is not particularly useful for long distance unlike radio. This meant that
we would have to greatly decrease the distances that we were simulating. This is
fine and would still allow a perfectly reasonable simulation so long as we
increased the distances again in our computations while analyzing the data. The
other problem that we ran into is the fact that the wireless networking would
quickly require us to go well outside the scope of our project. Determining the
gain and energy required to configure our satellites to send and receive
messages correctly is more of an engineering problem that we were not interested
in solving.

After settling on using point-to-point, UDP was easy to get working but TCP
proved to be very challenging. Our initial implementation of our topology seemed
to have an issue where messages could be routed in one direction but not the
other. This meant that when TCP tried to establish a connection it would always
fail. It still is not entirely clear what was configured incorrectly, but after
changing how we were creating channels, TCP was finally working. We ran into
more issues with TCP though, when we started adding delay to the channels. It
turns out that NS-3 sockets have a rather low time-to-live and there do not seem
to be any global options for configuring sockets. We were left with two options:
implement TCP from scratch using custom-built sockets for each node in the
network or ensure that the messages that we do send have lower delay than the
TTL for our sockets. We opted for the latter. To make sure that TCP messages
could successfully ACK, we found a value that we could divide all times by to
make sure that almost every message that could feasibly be sent would be
received before its TTL expired. This does not change any distances or error
rates and effectively only changes the unit we use for time from $s$ to
$\frac{s}{26}$ which has an easy conversion back to $s$. Finally, we implemented
a version of our program using the New Reno variant of TCP. Thankfully, NS-3
made this easy for us. It worked perfectly after assigning a configuration value
before setting up our communication channels.

\printbibliography{}

\end{document}
